\setlength{\LTpost}{0mm}
\begin{longtable}{l|rrrrr}
\caption*{
{\large Table S6. Pooling effect sizes competence, warmth, and callback for the categories race and gender}
} \\ 
\toprule
\multicolumn{1}{l}{} &  & \multicolumn{2}{c}{95\% CI} &  &  \\ 
\cmidrule(lr){3-4}
\multicolumn{1}{l}{} & estimate & lower & upper & p-value & SE \\ 
\midrule
\multicolumn{6}{l}{competence\textsuperscript{1}} \\ 
\midrule
black\_c\textsuperscript{2} & -11.52 & -23.74 & 0.71 & 0.06 & 3.84 \\ 
female\_c\textsuperscript{3} & -3.07 & -9.56 & 3.42 & 0.32 & 2.91 \\ 
\midrule
\multicolumn{6}{l}{warmth\textsuperscript{1}} \\ 
\midrule
black\_w\textsuperscript{2} & -6.72 & -19.19 & 5.76 & 0.19 & 3.92 \\ 
female\_w\textsuperscript{3} & 2.88 & -4.39 & 10.16 & 0.40 & 3.27 \\ 
\midrule
\multicolumn{6}{l}{callback\textsuperscript{4}} \\ 
\midrule
black\textsuperscript{5} & 0.79 & -0.51 & 0.04 & 0.07 & 0.09 \\ 
female\textsuperscript{6} & 1.02 & -0.03 & 0.06 & 0.36 & 0.01 \\ 
\bottomrule
\end{longtable}
\begin{minipage}{\linewidth}
\textsuperscript{1}The meta-analytical involves the inverse variance method and a restricted maximum-likelihood estimator for tau\textasciicircum{}2. The Q-Profile method was used to compute the confidence interval of tau\textasciicircum{}2 and tau, and a Hartung-Knapp (HK) adjustment was applied for the random effects model, with degrees of freedom set to 10.\\
\textsuperscript{2}k=4 studies, o=687 observations.\\
\textsuperscript{3}k=11 studies, o=816 observations.\\
\textsuperscript{4}The Mantel-Haenszel method was used to calculate the overall effect size, with the Paule-Mandel estimator used to estimate the between-study variance (tau\textasciicircum{}2). A random-effects model was employed with the Hartung-Knapp (HK) adjustment to account for potential bias due to small sample sizes. The model had 1 degree of freedom (df = 1).\\
\textsuperscript{5}k=4 studies, o=89872 observations.\\
\textsuperscript{6}k=4 studies, o=143860 observations.\\
\end{minipage}

